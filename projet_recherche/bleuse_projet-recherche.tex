%!TEX encoding = UTF-8 Unicode

%\documentclass[11pt,draft,french]{article}	% mettre la langue principale en dernier
\documentclass[11pt,final,french]{article}	% mettre la langue principale en dernier


%------------------------------------------------------------------------------+
%    ENCODAGE                                                                  |
%------------------------------------------------------------------------------+

% format de sauvegarde -----------------
\usepackage[utf8]{inputenc}
\usepackage[T1]{fontenc}
\usepackage{textcomp}


%------------------------------------------------------------------------------+
%    MISE EN PAGE DU DOCUMENT                                                  |
%------------------------------------------------------------------------------+

% taille des pages (A4 serré) ----------
\usepackage{geometry}
\geometry{a4paper}
\geometry{inner=27.5mm,outer=27.5mm,top=27.5mm,bottom=27.5mm} % change marges par défaut

% i18n ---------------------------------
\usepackage{babel}
\usepackage{isodate}

% figures ------------------------------
\usepackage{graphicx}
\graphicspath{{./img/}}

% citations ----------------------------
\usepackage{cite}
\bibliographystyle{acm}

% support .pdf étendu ------------------
% /!\ laisser hyperref en dernier package chargé
\usepackage{hyperref}

\hypersetup{
%	hidelinks,								% liens cliquables sans cadre
	pdfinfo={								% méta-informations supplémentaires
		Title={Projet de Recherche},
		Subject={Abstractions Qualitatives pour l'Ordonnancement Exascale},
		Author={Raphaël Bleuse},
	}
}


%------------------------------------------------------------------------------+
%    MACROS                                                                    |
%------------------------------------------------------------------------------+

% macros fournies par des packages -----
\usepackage[obeyDraft]{todonotes}
\let\todoraw\todo
\renewcommand{\todo}[1]{\todoraw[inline]{#1}}

% macros génériques --------------------
\newcommand{\cleardoubleemptypage}{\newpage{\pagestyle{empty}\cleardoublepage}}
\newcommand{\shell}[1]{\texttt{#1}}

% macros spécifiques -------------------
\newcommand{\cf}{cf.}
\newcommand{\cad}{c.-à-d.}
\newcommand{\etal}{\emph{et al.}}
\newcommand{\xkaapi}{XKaapi}
\newcommand{\dada}{\textsc{Dada}}
\newcommand{\heft}{\textsc{Heft}}
\newcommand{\moebus}{\textsc{Moebus}}


%------------------------------------------------------------------------------+
%    DOCUMENT                                                                  |
%------------------------------------------------------------------------------+

\title{
	Projet de Recherche \\
	\large Abstractions Qualitatives pour l'Ordonnancement \emph{Exascale}
}
\author{
	Raphaël \textsc{Bleuse}
	({\small \href{mailto:raphael.bleuse@imag.fr}{raphael.bleuse@imag.fr}})
}
\date{\today}

\begin{document}

\maketitle

\todo{ajouter résumé}

Le prochain jalon que la communauté du calcul haute performance s'est posé est
l'\emph{exascale} (\(10^{18}\)~Flop/s\footnote{Flop/s: opération flottante par
seconde}).
%
Atteindre une telle puissance de calcul tout en maintenant les coûts de
construction et d'exploitation raisonnables nécessite d'une part de changer la
conception des machines et d'arriver à utiliser au mieux les ressources
disponibles d'autre part.

Les machines hybrides -- constituées d'unité de calcul plus ou moins
spécialisées pour certaines tâches (calculs, entrées/sorties, …) -- sont une
approche envisagée par les constructeurs pour augmenter les performances et
l'efficacité énergétique des nouvelles architectures.

Diminuer les coûts peut aussi se faire en réorganisant les réseaux de
communication.
%
Cette réorganisation peut se faire via deux axes orthogonaux : via la
conception et l'étude de nouvelles topologies ou via la fusion des réseaux de
communication inter processus et d'entrées/sorties.
%
Néanmoins, cette multiplication des types de ressources de calcul, ainsi que la
fusion des différents flux réseau constitue un \textbf{défi pour les couches
logicielles chargées de gérer les ressources}.

\

Dans le cadre de mon doctorat, nous nous sommes intéressé au problème de
l'ordonnancement d'applications sur les machines parallèles.
%
Une application est représentée comme un ensemble de tâches élémentaires à
effectuer : trouver un ordonnancement revient à déterminer pour chaque tâche où
et quand s'exécuter.
%
Les impacts des évolutions susmentionnées sur l'ordonnancement se font à
plusieurs échelles : les évolutions architecturales des ressources de calcul
ont un impact à grain fin (\cad{} au sein d'un unique nœud de calcul) alors que
les évolutions des réseaux se font ressentir au niveau de la machine complète.
%
Au vue de la taille des machines actuelles, utiliser une description
détaillée et quantitative des machines ne semble pas raisonnable et ne passera
vraisemblablement pas à l'échelle.


\section{Principaux résultats obtenus}

Le début de mon doctorat a été consacré à l'étude d'ordonnancements à grain
fin.
%
Nous nous sommes intéressé à l'ordonnancement de tâches séquentielles
indépendantes au sein d'un nœud composé de deux types de ressources de calcul.
%
Nous avons proposé une notion d'\emph{affinité} comme indicateur qualitatif
pour rendre compte du degré de liaison entre les différentes tâches d'une
application parallèle.
%
L'idée sous-jacente était que des tâches en fortes interactions puissent être
allouées sur des ressources voisines.
%
Ces travaux ont été présentés pendant la conférence
Euro-Par~2014\cite{BleuseR2014Scheduling}.
%
Dans la même veine, nous avons implémenté des algorithmes d'ordonnancement
proposant de meilleures garanties au prix d'une complexité plus grande.
%
Confirmant nos intuitions, le surcout de calcul induit est trop important et
dégrade les performances globales.
%
Ces travaux ont conduit à une publication dans le journal
CCPE\cite{BleuseR2015Scheduling}.

La suite naturelle de ces travaux a été l'extension du modèle d'exécution des
tâches : les tâches sont considérées \emph{moldables} sur un type de ressource
de calcul.
%
Le problème a été abordé à l'aide d'un algorithme reposant sur un programme
linéaire rapide, mais dont la complexité en pire cas est inconnue.
%
Nous avons aussi proposé un algorithme relaxé de faible complexité.
%
Suite à une première évaluation encourageante, un article en révision favorable
et décrivant ces travaux vient d'être soumis au journal
TPDS\cite{BleuseR2016Scheduling}.

\

Dans le cadre d'une collaboration au sein du partenariat inter laboratoire
JLESC\footnote{\emph{Joint Lab. for Extreme Scale Computing} -- \cf{}
\url{http://publish.illinois.edu/jointlab-esc/}}, j'ai eu accès à Blue~Waters
-- la machine opérée par le NCSA\(\!\)\footnote{\emph{National Center
for Supercomputing Applications} -- \cf{} \url{http://www.ncsa.illinois.edu/}}
%
La topologie réseau particulière de Blue~Waters présente des défis pour
l'ordonnancement à l'échelle de la machine~\cite{EnosJ2014Topology}.
%
À partir des traces d'exécution de la machine, nous avons proposé une
modélisation des machines prenant en compte les contraintes imposées par les
topologies hétérarchiques~\cite{BleuseR2016Convex}.
%
Ces topologies ont la particularité de proposer une vision de la machine
beaucoup moins hiérarchique que les topologies habituelles (\emph{fat tree},
par exemple).
%
Par exemple, les nouvelles topologies réseaux proposées proposent des
hétérarchies locales connectées dans une hiérarchie de faible
profondeur~\cite{BestaM2014Slim,TuncerO2015PaC}.
%
Les contraintes dérivent alors principalement de l'agencement des nœuds
hétérogènes au sein de la topologie.
%
La prise en compte des contraintes n'est pas faite de manière quantitative,
mais au travers de propriétés géométriques : les allocations sont contraintes à
être convexes.

\section{Axes de recherche prévus}

Le modèle théorique évoqué ci dessus a été bien reçu par la communauté.
%
Deux pistes à explorer dans le cadre de ce modèle ont été identifiées.

L'entrelacement des flux d'entrée/sortie des différentes applications au sein
d'un réseau perturbe les performances.
%
La convexité des allocations permet de limiter ces interférences.
%
Obliger les applications à se situer près des nœuds d'entrée/sortie peut
permettre d'éliminer complètement ces interactions néfastes.
%
Quelles stratégies peuvent être mises en place pour ajouter ces contraintes à
un cout raisonnable ?
%
Quel est l'impact de ces contraintes supplémentaires sur le taux d'utilisation
des machines ?

D'autre part, pour un certain nombre de ressources demandées par une
application, il existe plusieurs formes convexes qui répondent aux besoins.
%
Une piste que nous souhaitons explorer est d'étudier quel choix de forme permet
de mieux exploiter les machines.
%
Dans la même optique, nous souhaiterions étudier si allouer plus de ressources
que demandé permettrait de simplifier l'étape d'allocation sans sans perdre
trop de puissance de calcul.


\bibliography{publications,references}

\end{document}
