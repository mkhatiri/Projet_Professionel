%!TEX encoding = UTF-8 Unicode

%\documentclass[11pt,draft,french]{article}	% mettre la langue principale en dernier
\documentclass[11pt,final,french]{article}	% mettre la langue principale en dernier


%------------------------------------------------------------------------------+
%    ENCODAGE                                                                  |
%------------------------------------------------------------------------------+

% format de sauvegarde -----------------
\usepackage[utf8]{inputenc}
\usepackage[T1]{fontenc}
\usepackage{textcomp}


%------------------------------------------------------------------------------+
%    MISE EN PAGE DU DOCUMENT                                                  |
%------------------------------------------------------------------------------+

% taille des pages (A4 serré) ----------
\usepackage{geometry}
\geometry{a4paper}
\geometry{inner=27.5mm,outer=27.5mm,top=27.5mm,bottom=27.5mm} % change marges par défaut

% i18n ---------------------------------
\usepackage{babel}
\usepackage{isodate}

% figures ------------------------------
\usepackage{graphicx}
\graphicspath{{./img/}}

% citations ----------------------------
%\usepackage{cite}
\bibliographystyle{acm}

% support .pdf étendu ------------------
% /!\ laisser hyperref en dernier package chargé
\usepackage{hyperref}

\hypersetup{
%	hidelinks,								% liens cliquables sans cadre
	pdfinfo={								% méta-informations supplémentaires
		Title={Projet de Recherche},
		Subject={ordonnancement de tâche sur plate-forme hétérogènes },
		Author={Mohammed KHATIRI},
	}
}


%------------------------------------------------------------------------------+
%    MACROS                                                                    |
%------------------------------------------------------------------------------+

% macros fournies par des packages -----
\usepackage[obeyDraft]{todonotes}
\let\todoraw\todo
\renewcommand{\todo}[1]{\todoraw[inline]{#1}}

% macros génériques --------------------
\newcommand{\cleardoubleemptypage}{\newpage{\pagestyle{empty}\cleardoublepage}}
\newcommand{\shell}[1]{\texttt{#1}}

% macros spécifiques -------------------
\newcommand{\cf}{cf.}
\newcommand{\cad}{c.-à-d.}
\newcommand{\etal}{\emph{et al.}}
\newcommand{\xkaapi}{XKaapi}
\newcommand{\dada}{\textsc{Dada}}
\newcommand{\heft}{\textsc{Heft}}
\newcommand{\moebus}{\textsc{Moebus}}


%------------------------------------------------------------------------------+
%    DOCUMENT                                                                  |
%------------------------------------------------------------------------------+

\title{
	Projet de Recherche \\
	\large Ordonnancement de tâches sur plate-forme hétérogènes 
}
\author{
	Mohammed \textsc{Khatiri}
	({\small \href{mailto:mohammed.khatiri@inria.fr}{mohammed.khatiri@inria.fr}})
}
\date{\today}

\begin{document}

\maketitle

\todo{ajouter résumé}


%Ajouter ici une intro générale qui montre ce que tu as fait et ce que tu fais :



Après l'obtention de mon master à l'Université Mohammed Premier à Oujda (UMPO).
Je me suis engagé dans un sujet de recherche au sein de la même université sur l'ordonnancement
d'un ensemble de tâches périodiques de délai implicite sur un multiprocesseur temps réel.
Ce travail qui m'a fait découvrir le domaine de la recherche a été publié dans la conférence internationale 
AICCSA 2015 \cite{confDaoudiDKK15}.
J'ai aussi eu l'occasion d'avoire un postie de vacataire au sein de la même université,
puis, j'ai démarré un doctorat en co-tutelle avec Mostafa El Daoudi (UMPO) et Denis Trystram (UGA)
sur Ordonnancement de tâches sur plate-forme hétérogène.


L'ordonnancement des applications sur les plates-formes de calcul parall\`eles et distribu\'ees
de nouvelle g\'en\'eration est un probl\`eme crucial et difficile.
En particulier, ces environnements se complexifient avec le passage à l'\'echelle vers
des multi-coeurs de plus en plus gros, des hi\'erarchies m\'emoires locales toujours
plus profondes et l'h\'et\'erog\'en\'eit\'e des composants de calcul
(processeurs classiques avec acc\'el\'erateurs, clusters avec différentes topologies,
plus noeuds sp\'ecialis\'es dans les Entr\'ees/Sorties ou dans l'analytics, etc.).
Un des probl\`emes essentiels est la gestion et l'utilisation efficace
des ressources de calcul h\'et\'erog\`enes (CPU, GPU, ARM, FPGA) pendant l'exécution des applications. 

\section{Contexte du sujet de th\`ese}

Le problème principal de ma thèse est l'ordonnancement des applications
sur les nouvelles plates-formes de calcul.
Une application est représentée comme un ensemble de t\^aches ind\'ependantes (t\^aches parall\`eles),
ou avec d\'ependences (DAGs de tâches parallèles)
où des t\^aches qui communiquent entre eux via la m\'emoire.
L'ordonnancement consiste \`a d\'eterminer les processeurs sur lesquels
chaque t\^ache doit \^etre ex\'ecutée et l’ordre dans lequel les t\^aches d\'ebuteront
leur ex\'ecution dans le respect des contraintes de pr\'ec\'edence.
Dans la littérature, il y existe deux types d'ordonnancement: (1)~hors ligne (ou bien statique)
qui ordonnance toute l'application avant le d\'ebut de l'ex\'ecution.
(2)~En-ligne (dynamique) où chaque t\^ache est ordonnanc\'ee
sur l'un des processeurs de la plate-forme d\`es qu'elle arrive dans le syst\`eme.
%ou toutes ces t\^aches pr\'ec\'edentes on finit leur ex\'ecution. 

%Là, il faut décrire le résultat technique en quelques lignes... le Résulata est dans la partis Principaux travaux


\subsection{Principaux travaux effectués}

Mes travaux de recherche concernent notamment l'étude de l'ordonnancement
des différents types d'applications sur les plates-formes hétérogènes.
L'hétérogénéité peut apparaître au niveau du calcul (CPU avec différent vitesse, CPUs-GPUs)
ou au niveau des communications explicites. 
Nous avons travaillé sur trois sujets complémentaires qui sont détaillés ci-dessous.

\begin{enumerate}
\item 
Nous nous sommes intéressés tout d'abord à l'étude de l'ordonnancement
d'un ensemble de tâches dans un cluster où la communication entre les processeurs est implicite.
Nous utilisons l'algorithme du vol de travail (en ligne) où chaque processeur gère sa liste de tâches,
quand un processeur devient inactif, il choisit aléatoirement un autre processeur pour voler du travail.
Le problème devient difficile quand le temps de vol de tâches prend un temps significatif qui perturbe
l'ordonnancement (processeurs inactifs mais entrain d'attendre des tâches).
Nous avons réussi à analyser mathématiquement (en utilisant les fonctions potentielles)
et l'impact de la communication sur l'ordonnancement.  
La première partie de ces travaux a été sélectionné pour être présenté à la conférence ECCO 2018
\cite{Nicolas:khatiri:Denis:Fred:2018} à Freiburg en juin prochain.
Une version complète a été soumis dans le journal ACM TOPC en début 2018.

\item 
La deuxième partie de ma thèse se base sur l'ordonnancement sur 2 GPUs (hétérogénéité de calcul)
pour un problème particulier (page-rank).
Le défit de ce problème est de faire l'équilibrage de charge pour calculer le produit matrice vecteur (base du page-rake).
Nous avons implémenté des algorithmes d'ordonnancement en utilisant le vol de travail adapté au GPUs pour faire l'équilibrage de la charge entre les deux GPUs.
La moitié de ce travail est commencé a l'université Université de Caroline du Nord à Charlotte (USA)
où j'ai passé deux mois à l'automne 2016.

\item La troisième partie où nous avons étudié un problème où l'hétérogénéité n'est plus simplement au niveau des calculs
qui consiste a l'ordonnancement des Microservices sur un ensemble de machines.
Un microservice est un petit composant de bibliothèque qui effectue une partie d'une application
constituée de plusieurs microservices .  
Les microservices sont conteneurisés (c'est-à-dire, ils sont isolés du reste d'application) dans des \textit{pods} avec différentes configurations.
Dans un premier temps, nous avons terminé la modélisation du problème,
en proposant des algorithmes simples pour gérer efficacement les ressources
et améliorer la qualité de service. 
Ce travail est le résultat d'un séjour de deux mois à l'université USP de Sao Paulo au Brésil.
\end{enumerate}

\subsection{Projet scientifique}

La suite naturelle de mes travaux a été l'extension du même problème de work stealing sur les plates-formes hiérarchiques,
où le temps de vol a l'intérieur d'un cluster est beaucoup plus petit qu'a l'intérieur.
Nous avons implémenté des algorithmes qui prennent on considération
la probabilité de vol (voler à l'intérieure est préférable).
Les résultats de simulation prouvent que le choix de la probabilité impact
les performances et aussi le mouvement de données dans la plate-forme.
Une première version sera soumis à la conférence HiPC 2018.
L'analyse mathématique de ce problème est en cours.

\bibliography{publications,references}

\newpage

\section{Activités d'enseignement}
Après mon Master, j'avais l'occasion d'enseigner en tant que vacataire.
J'ai fait 20h de TP de langage C à la faculté des sciences d'Oujda.
En même temps, j'ai été chargé d'un cours et encadrement de projet
au sein IT Training Center à OUJDA. 
Depuit 2016, j’ai commencé à enseigner aussi en France en étant vacataire à l'UGA (130h en 2ans),
j'ai enseigné l'algorithmique et les langage (C, java, Ocaml).
Plus que l'enseignement j'avais l'ocassion pour participer à la préparation
de la matière enseignée (sujet de TP et sujet d'examen),
faire la suivi des étudiant (contrôle continu), la correction de leurs devoirs.

Les matières d'enseignement en détailles :



\begin{description}
    \item [\textbf{ 2017 -- 2018 *} ]  20h de TP en algorithmique et programmation par objets en licence L2 MIASHS avec Mr. Jérôme David

        \par    contact :  \href{mailto:jerome.david@univ-grenoble-alpes.fr  }{ jerome.david@univ-grenoble-alpes.fr }

    \item [ \hspace{2.3cm} \textbf{*} ]  15h de TP en initiation à l'informatique et l'algorithme en licence L1 MIASHS avec Professeur Benoit Lemaire
        \par    contact :  \href{mailto:benoit.lemaire@univ-grenoble-alpes.fr}{benoit.lemaire@univ-grenoble-alpes.fr}
     \item [ \hspace{2.3cm} \textbf{*} ]  9.5h de Programming language and compiler design en Master M1 MOSIG avec Professeur Denis Trystram
        \par contact : \href{mailto:denis.trystram@imag.fr}{denis.trystram@imag.fr}
        
    \item [\textbf{ 2016 -- 2018 *} ]  66h (45h/21h)  de TP et TD en algorithmique et Programmation Fonctionnelle avec Professeur LAURENT Mounier pour Licence L1 à DLST. 
	    \par    contenus : Types, Récursivité, Fondements mathématiques, Modélisation et Algorithmique, Programmation avancée et langage Ocaml. 
        \par    contact :  \href{mailto:laurent.mounier@univ-grenoble-alpes.fr}{laurent.mounier@univ-grenoble-alpes.fr}
%
    \item [\textbf{ 2015 -- 2016 *} ]  18h de TP en algorithmique avancée avec Professeur Denis Trystram à ENSIMAG  
        \par contenus : analyses asymptotiques, diviser pour régner et programmation dynamique
        \par contact : \href{mailto:denis.trystram@imag.fr}{denis.trystram@imag.fr}



    \item [\textbf{ 2014 -- 2015 *} ]%
         20h de TP en algorithmique et programmation en C a la faculté des scince de OUJDA avec Madame Amina Yahya
	     contact :  \href{mailto:aminayahia@yahoo.fr}{aminayahia@yahoo.fr}
%
\end{description}

%\bibliography{publications,references}

\end{document}
